\chapter {Details of the compiler package}
\section {The package}
The following files and directories should be present in the compiler
package:
\begin {list}{-}
{\setlength \leftmargin {7em}
\setlength \labelwidth {6em}
\setlength \labelsep {1em}}
\item [{\tt Copyright}] Copyright notice.
\item [{\tt Warranty}] Warranty notice.
\item [{\tt Install}] Installation notes.
\item [{\tt History}] History of changes to the compiler package.
\item [{\tt Planned}] Planned changes to the compiler package.
\item [{\tt Makefile}] Makefile to make all libraries and the compiler.
\item [{\tt libalib}] An example user extension of the standard library.
\item [{\tt bin}] Directory to install binaries (initially empty).
\item [{\tt buglist}] List of known bugs.
\item [{\tt doc}] Directory containing the source files of this document.
\item [{\tt examples}] Directory containing the examples in this
document.
\item [{\tt gen}] Directory containing the source files of the compiler
and skeleton generator.
\item [{\tt include}] Directory to install header files (initially empty).
\item [{\tt lib}] Directory to install libraries (initially empty).
\item [{\tt libXedt}] Directory containing the source files of the
\Xelf window system support library. In particular it provides
two widgets that are used by syntax directed editors.
\item [{\tt libebs}] Directory containing the source files of the
\EAG basic support library.
\item [{\tt libeag}] Directory containing the source files of the
\EAG runtime system.
\item [{\tt libedt}] Directory containing the source files of the
runtime system necessary for syntax directed editors.
\end{list}
\section{Installation}
To install the \EAG compiler and its libraries, edit the {\tt Makefile}
in the top directory and adapt the variables {\tt IDIR}, {\tt LDIR},
{\tt BDIR}, {\tt DCFL} and {\tt X11TREE} to suit your local system
(The makefile contains two possible suggestions, depending whether you
wish to install the compiler locally or system wide). Then enter
{\tt make} to create and install the necessary libraries and the
compiler.

The software has been tested on the following platforms:
\begin {list}{-}{}
\item Sun SPARCstation-20 under Solaris 2.7
\item Sun 4/670 under SunOs 4.1.4
\item Sun 3/80 under SunOs 4.1.1
\item HP 9000/735 under HP-UX 9.0.5
\item i686 PC under Linux 2.2.12
\end {list}
\section{Compiler options}
The syntax to use the compiler is:
\begin{verbatim}
   eag-compile [flags] [name]
\end{verbatim}
The default extension of \EAG is {\tt eag}. Preferably, the name
of the file should be given without this extension. The compiler will
generate a \C file with name \verb+<name>_<type>.c+ where
\verb+<name>+ is the name given as argument and
\verb+<type>+ is either {\tt topdown} or {\tt leftcorner},
depending on the type of parser wanted.
When no input file is given, the compiler will read a grammar from
standard input giving it {\tt stdin} as name.

The meaning of the flags is as follows:
\begin {list}{-}{}
\item [{\tt -h}] Provide a short help.
\item [{\tt -V}] Show the current version of the compiler package.
\item [{\tt -v}] Verbose mode. Provide an overview of the passes
of the compiler.
\item [{\tt -fv}] Full Verbose mode. Provide an overview of the passes
and dump intermediate analysis results on standard error. This option
is provided mainly for debugging purposes, since it will create
large amounts of output.
\item [{\tt -p name}] Include {\tt name.eag} as extra prelude.
This option is mainly used to include user extensions of the
standard library.
\item [{\tt -v1}] Accept \EAG version 1.6 syntax and semantics.
This is a compatibility flag.
\item [{\tt -ed}] Generate a syntax directed editor.
\item [{\tt -td}] Generate a topdown parser.
\item [{\tt -lc}] Generate a leftcorner parser.
\item [{\tt -lr}] Generate a lr parser (not yet implemented).
\item [{\tt -nl}] Do not generate lookahead code. Setting this flag
will degrade the performance of the generated parsers. However, the
error messages generated when a parse can not be found will improve
much.
\item [{\tt -gl}] Generate code to report error messages upon
lookahead failure. This will improve the error messages generated
when a parse can not be found.
\item [{\tt -ic}] Generate code to interface with \Cns.
\item [{\tt -il}] Do not code any actions on the syntax tree for the
syntax rule {\tt layout}. This rule should comply to the rules
given in section 5.3. Setting this flag will improve the execution
time of generated parsers by some 20\%. This flag is implied by the
{\tt -ed} flag.
\item [{\tt -pp}] Generate code to parse placeholders. This flag is
implied by the {\tt -ed} flag.
\item [{\tt -it}] Generate code to show {\em indirect templates} too. An
indirect template is a template derived from a template that consists
of a single typed placeholder by replacing this placeholder by one
of its corresponding templates.
\item [{\tt -D}] Generate code to dump the parse tree when a
succesful parse is found.
\item [{\tt -M}] Generate code that dumps the number of {\em matches}
found while parsing. A match is the succesful recognition of a terminal.
This number is often used by linguists to compare various implementations
of \EAG or \AGFLns.
\item [{\tt -T}] Generate code to trace the execution of a parser.
\item [{\tt -qc}] Generate code to check the proper behaviour
of the continuation stack. This flag is solely used for debugging
purposes.
\end {list}
\section {Scripts}
Four shell scripts are generated by the installation procedure.
Their usage is as follows:
\begin{verbatim}
   <script> <name>_<type> [options]
\end{verbatim}
where \verb+<script>+ is one of these four scripts, \verb+<name>+
and \verb+<type>+ are as above. The options are only used for user
extensions. These scripts are used to compile and link generated
parsers or editors with the proper libraries. They will result
in an executable with the name \verb+<name>_<type>+. The four
scripts and their functions are:
\begin {list}{-}
{\setlength \leftmargin {5em}
\setlength \labelwidth {4em}
\setlength \labelsep {1em}}
\item [{\tt eagcc}] Compile and link a generated transducer.
\item [{\tt eagccl}] Only link a generated transducer.
\item [{\tt eaged}] Compile and link a generated syntax directed editor.
\item [{\tt eagedl}] Only link a generated syntax directed editor.
\end {list}
\section {The standard library}
Before a grammar is parsed by the \EAG compiler a prelude containing
builtin definitions is parsed ({\tt stddefs.eag}). The contents
of this file are as follows:
\begin{verbatim}
   # standard predefines
   # meta rules
   $ int :: int, recognizer, may produce empty.
   $ string :: string, recognizer, may produce empty.
   $ tuple :: tuple, recognizer, never produces empty.
   $ nonlat :: nonlat, recognizer, may produce empty.
   $ nil :: tuple, single, never produces empty.

   # standard meta rules
   empty ::.
   nlcr :: "\n".
   quote :: "\"".
   back slash :: "\\".
   tab :: "\t".

   # hyper rules
   empty: .
   nlcr: "\n".
   quote: "\"".
   back slash: "\\".
   tab: "\t".

   # semi predicates: they need not be delayed
   $ end of sentence: semipredicate.
   $ column (int>): semipredicate.
   $ row (int>): semipredicate.

   # predicates
   $ equal (>nonlat, >nonlat): predicate.
   $ not equal (>nonlat, >nonlat): predicate.

   $ minus (>int, >int, int>): predicate.
   $ mul (>int, >int, int>): predicate.
   $ div (>int, >int, int>): predicate.
   $ power (>int, >int, int>): predicate.

   $ int to string (>int, string>): predicate.
   $ string to int (>string, int>): predicate.
   $ uppercase (>string, string>): predicate.

   $ not in reserved word list (>string, >tuple): predicate.
   $ dump affix (>nonlat): predicate.
\end{verbatim}
Definitions that do not start with a dollar sign are merely included
and thus hardcoded in generated parsers. For definitions that do start
with a dollar sign actual code is present in the \EAG runtime library.

For a meta rule in the last category the type, kind and emptiness is
specified (The type is one of the basic types. The kind is
either {\tt single} or {\tt recognizer}. A meta rule may produce
empty or not).

A rule that is coded in the runtime library is either a predicate
or a semipredicate. Its formal left hand side is specified and
typed and its kind is specified.
\section {User extensions}
It is possible for users to extend the standard library with their
own libraries. The first step in making a private library is
writing its specification which should comply to the same syntax
as given for the standard library. Such a specification should
have {\tt eag} as its extension. An example of such a specification
can be found in the example extension library ({\tt libalib}) in the
compiler package by the name {\tt alib.eag}:
\begin{verbatim}
   $ make empty field (>int, >int): predicate.
   $ take (>int, >int): predicate.
   $ is empty (>int, >int): predicate.
   $ show field: semipredicate.
\end{verbatim}

The second step to take is to apply the skeleton generator
(\verb+eag-skel+), which is also installed by the installation
procedure, to this specification:
\begin{verbatim}
   eag-skel <name>
\end{verbatim}
This will result in two files namely \verb+<name>.h+ and
\verb+<name>.c+. The latter gives a template in which the user
only has to fill in the \C code that implements the actual
specification. The code to needed to delay the execution
of predicates is already coded by the skeleton generator. Before
taking this step it may be very useful to read the file
{\tt ds.h} in the directory {\tt libeag} of the package
which describes the datastructures used by the
\EAG runtime system. It is also advisable to inspect the file
{\tt stddefs.c} in the same directory to observe how the
standard library is implemented.

After filling in the actual code one must compile this \C file
and construct a small library called \verb+lib<name>.a+ for it.
This library should be installed in the {\tt lib} directory
of the package and the files \verb+<name>.eag+ and \verb+<name>.h+
should be placed in the {\tt include} directory. Compilation
of an \EAG using this user defined library needs adding the flag
\verb+-p <name>+ for the compilation and the flag \verb+-l<name>+
for the \C compilation and linking. For example, the file
{\tt cactus.eag} in the example user extension library
directory {\tt libalib} may be compiled and linked by entering the
following two commands:
\begin{verbatim}
   eag-compile -p alib -td cactus
   eagcc cactus_topdown -lalib
\end{verbatim}
This example also shows the usage of \EAG in nondeterministic programming.
