\chapter {Introduction}
This book intends to introduce the Extended Affix Grammar
formalism \cite{watt} and its compiler. The Extended Affix Grammar
formalism, or shortly \EAGns, is a formalism for describing both
the context free and the context sensitive syntax of languages.
\EAG is a member of the family of two-level grammars. They are
very closely related to two-level van Wijngaarden grammars
\cite {revised report}. To a description in \EAG the \EAG compiler
may be applied, generating either a recognizer or a transducer or
a translator or a syntax directed editor for the language described.

We will not discuss the exact details of the formalism, but we will
pay more attention to the ideas behind the language and
elucidate the language concepts by many examples. All these
examples can be found in the example subdirectory ({\tt examples})
of the compiler package. We will also discuss the use of the
compiler and the options that may be used to steer the
generation process. We will assume that the reader has
some basic knowledge of context free grammars.

In the following chapters we will use the {\tt typewriter} font to
indicate examples in \EAGns. In chapter 2 we will
discuss context free descriptions in \EAGns. Chapter 3 will 
introduce the second level of the formalism.
The next chapter introduces meta rules, which may impose a context free
grammar on the second level of the formalism.
In chapter 5 we will see how the compiler may be used for
generating a syntax directed editor from an \EAG description.
Finally, chapter 6 indicates the direction of future research.

There are 3 appendices. Appendix A gives some more details
of the package (It is useful to read the first two sections if you
wish to install the package). Appendix B gives a description of the
\EAG formalism in \EAGns. Finally the last appendix contains some
references.

This is the first version of a new major release (2.0) of the
\EAG compiler. As such, it is quite experimental and bugs may
turn up. Bug reports may therefore be sent to marcs@cs.kun.nl,
so that these bugs may be corrected in later versions.
\section*{Copyright Notice}
The copyright of the \EAG compiler package and its documentation
is held by the Computing Science Department of the University of
Nijmegen. Permission is hereby granted to make and distribute copies
of the source code and the documentation, provided the copyright
notice is preserved on all copies.
\section*{Warranty}
Because the \EAG compiler package is licensed free of charge, we
provide absolutely no warranty, to the extent permitted by
applicable law. The University of Nijmegen provides the \EAG
compiler package ``as is" without warranty of any kind.
\section*{Distribution}
The \EAG compiler package is available by ftp at {\tt ftp.cs.kun.nl}
in the directory {\tt /pub/eag}.
