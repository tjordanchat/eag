\chapter {Describing context free languages}
\section {Basic syntax}
An \EAG consists of {\em syntax rules\/}, {\em affix production
rules} (also called {\em meta rules)} and an optional {\em start rule}.
In this chapter we will focus on the syntax rules. A syntax rule
consists of one or more {\em syntax rule elements}, separated by
semicolons, ending in a period. A syntax rule element consists of
a left hand side, a colon and one or more {\em alternatives}.
Alternatives are separated by semicolons. The left hand side of a
syntax rule element consists of a {\em nonterminal} identifying
the syntax rule. For the present, nonterminals are identifiers
consisting of letters and digits. Blanks may occur in nonterminals
but they are ignored. Thus {\tt even more elements} and
{\tt evenmoreelements} denote the same name.

An alternative consists of zero or more members, separated by commas.
A member can either be a {\em terminal}, a nonterminal, a {\em semiterminal}
or a {\em commit operator}.  A terminal is a sequence of characters,
enclosed between quotes. Semiterminals and the commit operator will
be discussed later. When reading an \EAGns, one can
read the colon as ``is defined as", the semicolon as ``or",
the comma as ``followed by" and the period as ``end of definition".

The reader may perhaps wonder why there are two syntactic ways in
\EAG to indicate choice namely the semicolon between syntax rule
elements and the semicolon between alternatives. The explanation
of the reason behind this must unfortunately be postponed to chapter 3.
Consider the following example:
\begin{verbatim}
   smengels:
      subject, verb, ".".

   subject:
      article, noun.

   article: "the "; "a ".

   noun: "man ";
   noun: "dog ".

   verb: "sleeps"; "sits".
\end{verbatim}
As we can see this grammar specifies a language of the following 8
sentences:
\begin {list}{-}{\setlength {\itemsep}{0ex} \setlength {\parsep}{0ex}}
\item the man sleeps
\item a man sleeps
\item the dog sleeps
\item a dog sleeps
\item the man sits
\item a man sits
\item the dog sits
\item a dog sits
\end {list}
\EAGs must obey the following context conditions:
\begin {itemize}
\item
All left hand side identifiers of one syntax rule must be the same.
\item
The left hand side identifiers of different syntax rules must differ.
\item
For every nonterminal occurring in an alternative there must be
a syntax rule (or possibly a builtin syntax rule) with the same name.
\end {itemize}
We will call the application of a nonterminal on the right hand side a
{\em call} of that nonterminal. The start rule of an \EAG consists of
a single call of the starting syntax rule, followed by a period.
There are three (evident) context conditions concerning start rules:
\begin {itemize}
\item
For the nonterminal in the start rule, there must be a (possibly builtin)
syntax rule with the same name.
\item
More than one start rule is not allowed.
\item
If an \EAG does not specify a start rule, the textually first
syntax rule of the \EAG will be taken as start rule.
\end {itemize}
\section {A first use of the compiler}
Suppose we have the file {\tt smengels.eag} in the example directory
which contains the above syntax rules. We can compile this grammar by
entering the following two commands:
\begin{verbatim}
   eag-compile smengels
   eagcc smengels_leftcorner
\end{verbatim}
The first of these commands compiles the grammar into a recognizer 
in the programming language \Cns. The second one compiles this recognizer
into an executable, called {\tt smengels\_leftcorner} (Later we will
discuss why this name is given to the executable). If one executes
this file, it will read lines from standard input until an end of file is
encountered. This input will then be parsed by the recognizer, which
will then report whether the parse was succesfull or not.

Unfortunately, lines read from stdin in \Unix end in a newline.
Thus the recognizer will fail and report that no parse could be found.
We therefore adapt the above grammar in the following way (as given
in the file {\tt smengels2.eag} in the {\tt examples} subdirectory):
\begin{verbatim}
   smengels2.

   smengels2:
      subject, verb, ".", nlcr.

   subject:
      article, noun.

   article: "the "; "a ".

   noun: "man ";
   noun: "dog ".

   verb: "sleeps"; "sits".
\end{verbatim}
In this example we use the builtin syntax rule {\tt nlcr} to
specify the trailing newline.  If we compile the above example
we will see that the recognizer will not produce any error
message for grammatically correct input sentences. If
the recognizer can not find any correct parse it will generate
an error message including the farthest position in the input upto
which it could parse.

The \EAG compiler will automatically add a check in the start rule
to match the end of file at the end of the input. Thus it is
impossible to report a successfull parse, if any unconsumed
input remains.

The \EAG compiler generates {\em recursive backup} parsers.
Such a parser uses recursive procedures for parsing. 
The parser backtracks when the recognition of a symbol
fails.

Backtracking means undoing the effect of one or more calls the
parser has lastly done. If recognition of the whole input 
succeeds, the parser reports a ``successful parse" and then
backtracks. If at a certain point several steps are
possible they will be tried one after the other.

This means that a parser generated by the compiler will find all possible
parses of the input. Thus the compiler can handle {\em ambiguous}
grammars (A context free grammar is called ambiguous if there
exists a string with more than one distinct derivations according to that
grammar). This is one of the great advantages of the \EAG compiler
since many interesting languages are inherently ambiguous, such as
natural languages. The disadvantage of a backtracking parser is
its high (exponential) time complexity.
However the \EAG compiler codes several heuristics, such as lookahead,
in the generated parser to limit the amount of backtracking while parsing.
An example of an ambiguous grammar is the following (to
be found in the file {\tt examples/amb\_expr.eag}):
\begin{verbatim}
   start.

   start: expr, nlcr.

   expr: expr, "+", expr.
   expr: term.

   term: term, "*", term.
   term: id.

   id: "a"; "b"; "c"; "d"; "e"; "f".
\end{verbatim}
If we enter {\tt a+b+c} as input the generated parser will report
two successful parses. If the user wants to inspect the different
derivations, she must add the flag {\tt -D} when compiling the
grammar:
\begin{verbatim}
   eag-compile -D amb_expr
   eagcc amb_expr_leftcorner
\end{verbatim}
When parsing the input {\tt a+b+c} the parser will give the two
parse trees corresponding to the two derivations:
\begin{verbatim}
Dump of parse tree:              Dump of parse tree:
rootnode (0)                     rootnode (0)
 rule_start (11)                  rule_start (11)
  rule_expr (3)                    rule_expr (3)
   rule_expr (3)                    rule_expr (4)
    rule_expr (4)                    rule_term (13)
     rule_term (13)                   rule_id (5)
      rule_id (5)                   rule_expr (3)
    rule_expr (4)                    rule_expr (4)
     rule_term (13)                   rule_term (13)
      rule_id (6)                      rule_id (6)
   rule_expr (4)                     rule_expr (4)
    rule_term (13)                    rule_term (13)
     rule_id (7)                       rule_id (7)
  rule_nlcr (2)                    rule_nlcr (2)
\end{verbatim}
Such a dump of the parse tree will only show nonterminals (Terminals
are not stored in the parse tree built by the parser). The number
behind the nonterminal indicates which alternative
of a syntax rule is taken (All alternatives in an \EAG are given a unique
number by the parser generator. This number is shown in the dump).

Of course the grammar writer can make errors from time to time
when writing the grammar. Many of these will be flagged by the
compiler as syntax or identification errors. Sometimes it
is useful to obtain a trace of the execution of a parser.
One can obtain a trace by adding the flag {\tt -T} at
compilation time:
\begin{verbatim}
   eag-compile -T amb_expr
   eagcc amb_expr_leftcorner
\end{verbatim}
An example of this trace is the following piece of a trace
obtained from a parser for the above grammar for the input
{\tt a+b+c}:
\begin{verbatim}
   13 '\n': <red_id
   12 'c': <get_expr
   11 'c': <rule_expr
   10 '+': <red_expr
   10 '+': -rule_expr, alt 1
   10 'c': >rule_expr
   11 'c': >get_expr
   12 'c': -rule_id, alt 3
   12 '\n': >red_id
   :
   :
   18 '\n': >get_nlcr
   19 '\n': -rule_nlcr, alt 1
   19 '\e': >red_nlcr
   20 '\e': >red_start
   A parse was found
   20 '\e': <red_start
\end{verbatim}
The leftmost number of each line indicates the depth in recursion,
followed by the first unconsumed character in the input. The end
of file character is indicated by \verb+\e+, as can be seen in the
example. The remainder of the line indicates the parsing action
that is being executed at that point of the parse. This example also shows
the recursive backup nature of the parser (We can see that the parser
backups until the {\tt +} is reached again and then continue parsing
according to another alternative).
\section {Two kinds of parsers}
The \EAG compiler can generate two kinds of parsers namely
\begin {itemize}
\item topdown parsers
\item leftcorner parsers
\end {itemize}
Topdown parsers can be obtained by adding the flag {\tt -td} to
the command line when generating the parser. The kind of the parser
is added to the name of the grammar when generating the parser.
Suppose we want to generate a topdown parser for our {\tt smengels2}
example we have to enter:
\begin{verbatim}
   eag-compile -td smengels2
   eagcc smengels2_topdown
\end{verbatim}
resulting in an executable named {\tt smengels2\_topdown}.
Leftcorner parsers are obtained by adding the flag {\tt -lc} to
the command line at generation time. This flag is default.

There is a great difference between the two kinds of parsers.
Topdown parsers tend to be smaller in size, while leftcorner
parsers tend to be a little bit faster then topdown parsers.
However topdown parsers can not cope with {\em leftrecursive}
grammars. We call a grammar leftrecursive if there is a nonterminal
and a sentential form derived from that nonterminal in one or
more steps, starting with the same nonterminal. For such a grammar
the \EAG compiler will report an error message, when trying to
generate a topdown parser.
\section {Semiterminals}
As we can see from the provious example it is quite tedious
if we have to describe a syntax rule for a letter in the following way:
\begin{verbatim}
   letter: "a"; "b"; "c"; "d"; "e"; "f"; "g"; "h"; "i";
           "j"; "k"; "l"; "m"; "n"; "o"; "p"; "q"; "r";
           "s"; "t"; "u"; "v"; "w"; "x"; "y"; "z".
\end{verbatim}
A shorthand for the right hand side is therefore available,
called a semiterminal:
\begin{verbatim}
   letter: {abcdefghijklmnopqrstuvwxyz}.
\end{verbatim}
This set notation indicates that \verb+letter+ is one of the characters
in the set. If a set consists of a number of consecutive characters,
a more consise shorthand is available:
\begin{verbatim}
   letter: {a-z}.
\end{verbatim}
A semiterminal can have several features. Here is an overview:
\begin {list}{-}
{\setlength {\leftmargin}{7em}
\setlength {\labelwidth}{5em}
\setlength {\itemsep}{0.3ex}}
\item [{\tt \{....\}}]
matches a character of the set.
\item [{\tt \{....\}*}]
matches any sequence of zero or more characters of the set.
\item [{\tt \{....\}+}]
matches any sequence of one or more characters of the set.
\item [{\tt \{....\}*!}]
matches the longest sequence of zero or more characters of the set.
\item [{\tt \{....\}+!}]
matches the longest sequence of one or more characters of the set.
\item [{\tt \^{ }\{....\}}]
matches any (single) character not in the set.
\item [{\tt \^{ }\{....\}*}]
matches any sequence of zero or more characters not in the set.
\item [{\tt \^{ }\{....\}+}]
matches any sequence of one or more characters not in the set.
\item [{\tt \^{ }\{....\}*!}]
matches the longest sequence of zero or more characters not in the set.
\item [{\tt \^{ }\{....\}+!}]
matches the longest sequence of one or more characters not in the set.
\end {list}
Usually a character in a semiterminal (or a terminal) denotes itself.
Some special characters can also be denoted by using an escaping
mechanism similar to that used in \Cns:
\begin {center}
\begin {tabular}{|l|l|}
notation & denotes\\
\hline
\verb+\n+ & newline \\
\verb+\t+ & tab \\
\verb+\-+ & minus \\
\verb+\\+ & backslash \\
\verb+\"+ & quote \\
\verb+\}+ & right brace
\end {tabular}
\end {center}
This same escaping mechanism can also be used in terminals.

The following example is a grammar for expressions (It can be
found in the example directory as {\tt expr.eag}). It also
indicates more or less a preferred style of writing \EAGns s:
\begin{verbatim}
   start:
      layout, expr.
   
   expr:
      expr, plus symbol, term;
      expr, minus symbol, term;
      term.
   
   term:
      term, times symbol, factor;
      term, divides by symbol, factor;
      factor.
   
   factor:
      minus symbol, factor;
      plus symbol, factor;
      unsigned factor.
   
   unsigned factor:
      identifier;
      number;
      left parenthesis, expr, right parenthesis.
   
   identifier: {a-z}, {a-z0-9}*!, layout.
   
   number: {0-9}+!, layout.
   
   plus symbol: "+", layout.
   minus symbol: "-", layout.
   times symbol: "*", layout.
   divides by symbol: "/", layout.
   left parenthesis: "(", layout.
   right parenthesis: ")", layout.
   
   layout: { \n\t}*!.
\end{verbatim}
Observe the clear distinction between the rules of the
context free syntax and those of the lexical syntax.
\section {Grammar transformations}
In many cases the generated parsers may have bad runtime behaviour,
due to backtracking. However the grammar writer may improve
the runtime behaviour by transforming the underlying grammar
or by carefully rewriting it.

As mentioned earlier topdown parsers generated by the compiler are
smaller than leftcorner parsers. A grammar writer could therefore
aim at writing a grammar that is topdown parsable. Such a grammar
may not contain leftrecursive rules. Luckily (almost) every context
free grammar can be transformed into an equivalent context free
grammar that does not contain leftrecursive rules. Consider for
instance the following syntax rule with $k$ alternatives that are
directly leftrecursive and $l$ alternatives, which are not:
\begin{verbatim}
   A: A, rest alt1;
   A: A, rest alt2;
    :
   A: A, rest altk;
   A: altk+1;
    :
   A: altk+m.
\end{verbatim}
This syntax rule may be replaced by the following two syntax rules, which
are not leftrecursive:
\begin{verbatim}
   A: altk+1, B;
   A: altk+2, B;
    : 
   A: altk+m, B.

   B: rest alt1, B;
   B: rest alt2, B;
    :
   B: rest altk, B;
   B: .
\end{verbatim}
For example if we apply the above scheme to the syntax rule {\tt expr}
of the previous example we obtain the following rules:
\begin{verbatim}
   expr: term, more terms.

   more terms: plus symbol, term, more terms;
   more terms: minus symbol, term, more terms;
   more terms: .
\end{verbatim}
The above scheme works for every grammar that only contains
directly leftrecursive rules. If there are also indirectly
left recursive calls, a more general scheme must be used for
this transformation \cite{aho}.

Another technique worthwhile considering is that of leftfactoring.
By leftfactoring is meant the sharing of common prefixes of different
alternatives of a syntax rule. Consider for instance the following
syntax rule:
\begin{verbatim}
   asbc: ap, asbc, "b";
   asbc: ap, asbc, "c".
\end{verbatim}
By left factoring this rule we obtain:
\begin{verbatim}
   asbc: ap, asbc, rest asbc.

   rest asbc: "b"; "c".
\end{verbatim}
This technique can reduce the runtime complexity. In some cases it
may even reduce the exponential behaviour to a polynomial one. For
leftcorner parsers, nonterminals are automatically leftfactored by
the parser generator.
\section {The commit operator}
The commit operator may be used to indicate that whenever a topdown
parse succeeds beyond a certain point in an alternative, indicated by
this commit operator (\verb+->+), the alternatives succeeding this
alternative may be ignored by the parser. This operator is extremely
useful to speed up a topdown parser, whenever the grammar writer is sure
that two alternatives of a rule are mutually exclusive.

Thus, the commit operator in:
\begin{verbatim}
   apbc: ap1, ->, bc;
   apbc: ap2, bc.
\end{verbatim}
guarantees that whenever the parser succeeds beyond the member {\tt ap1},
the second alternative will not be tried.

The advantages of the commit operator are that it may be used to
speed up the generated topdown parsers or to make a generated parser
more deterministic. The disadvantage is that the order of the
alternatives of a syntax rule is now significant.

The commit operator is only implemented for topdown parsers. It is ignored
if a leftcorner parser is generated.
